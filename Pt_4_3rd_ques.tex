\documentclass{article}
\usepackage[utf8]{inputenc}

\author{path1380 }
\date{February 2018}

\begin{document}

\section*{Using Chebyshev polynomials / Monomials instead of Legendre polynomials}
Let a function f(x) be approximated by using Chebyshev polynomials upto degree q as basis functions. Let the grid comprise of intervals $\Omega_{i}$. So the function would have a representation
\begin{equation}
    f(x) = \sum_{k=0}^{q} c(k,i) T_{k}(r_{i}(x)) \hspace{1in} x \in \Omega_{i}
\end{equation}
where $r_{i}(x)$ be the linear mapping from $x \in \Omega_{i}$ to $r_{i} \in [-1,1]$. Thus, we would have
\begin{equation}
    \int_{\Omega_{i}} \frac{f(x) T_{l}(r_{i}(x)}{\sqrt{1 - (r_{i}(x))^{2}}} dx = \int_{\Omega_{i}} \sum_{k=0}^{q} c(k,i) \frac{T_{k}(r_{i}(x)) T_{l}(r_{i}(x))}{\sqrt{1 - (r_{i}(x))^{2}}} dx = c(l,i)
\end{equation}
Not only is the Chebyshev polynomial approximation objectively the best polynomial approximation technique for function approximation ($||Error|| < \frac{1}{2^{q-1}}$) but computing the coefficients is also pretty straightforward. So, this would be a pretty convenient way for approximating the function.\\

If we have a set of monomial basis functions, i.e. \{$1, x, x^{2}, \cdots x^{q}$\} then the function would be represented as 
\begin{equation}
    f(x) =  \sum_{k=0}^{q} c(k,i) x^{k} \hspace{1in} x \in \Omega_{i}
\end{equation}
We need to evaluate the function at q+1 points in each interval and solve a VanderMonde system of equations in each interval. Generally, this system is pretty ill-conditioned. We may have to switch over to Newton basis and use the divided difference formulae to calculate the interpolating polynomials using the points of function evaluation as interpolating points.\\

These are sample algorithms if we change the choice of basis functions.
\end{document}
